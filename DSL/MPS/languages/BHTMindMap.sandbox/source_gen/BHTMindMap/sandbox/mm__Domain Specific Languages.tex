\documentclass{article}
\usepackage[utf8]{inputenc}
\title{Einsendeaufgabe 9 – DSL}
\author{Jana Deutschlaender}
\date{February 2023}
\begin{document}
\maketitle
\paragraph{Die folgende Struktur wurde aus einer Mindmap der Sprache \emph{BHTMindMap} erstellt:Domain Specific Languages(DSL)}
\section{definition}
\subsection{focused on specific domain}
\subsection{language nature }
\subsection{limited expressiveness }
\section{categories}
\subsection{internal dsl}
\subsubsection{has grammar}
\subsubsection{do not use formal grammar}
\subsubsection{ability to use closure}
\subsubsection{bound by host language}
\subsection{external dsl }
\subsubsection{need language parser}
\subsection{language workbench}
\subsubsection{tools that allow to build languages and its environment}
\subsubsection{benefits}
\paragraph{freedom to build dsls}
\section{tools }
\subsection{xText }
\subsection{MPS}
\section{semantic model }
\subsection{state machine }
\subsection{interpretation}
\subsection{code generation }
\section{benefits }
\subsection{programmer productivity }
\subsection{express behaviour in a natural way to the domain}
\section{disadvantages}
\subsection{cost of building dsl }
\subsection{migration problem when business evolves and requires modified dsl}
\subsection{understanding dsl - what is a good dsl}
\section{life cycle }
\subsection{build dsl}
\subsection{evolve dsl}
\subsubsection{how do we migrate build system}
\subsection{impact on build system}
\subsection{language parser}
\subsection{testing}
\section{syntactic facade over complex framework }

\end{document}